%% 
%%	This is file 'beamer_sample.tex'
%%	according to an MPIDR's PowerPoint template (?)
%%	
%%	by Eric Naujoks
%%
%%	Problems, bugs and comments to 
%%	naujoks@demogr.mpg.de
%%

%%%%%%%%%%%%%%%%%%%%%%%%%%%%%%%%%%
%%	Praelegomena								%%
%%%%%%%%%%%%%%%%%%%%%%%%%%%%%%%%%%
%%	- Make sure that you use utf8-encoding for all your .tex-files!!! (TeXnicCenter since version 2.0)
%%	- TeXnicCenter update: MPIDR intranet > Hard- & Sortfware > Software > Script and text editors > TeXnicCenter

\documentclass[20pt,usenames,dvipsnames]{beamer}

\usepackage[ngerman,english]{babel}
\usepackage{tikz}
\usepackage[normalem]{ulem}
\geometry{paperwidth=10in, paperheight=7.5in}
\usepackage{animate}

\usepackage[utf8]{inputenc}

\usepackage[mpidr]{./mpidr/beamerthemeMPIDR}
%\usefonttheme{serif}
%\newcolumntype{C}[1]{>{\centering\let\newline\\\arraybackslash\hspace{0pt}}m{#1}}
%\newcommand*{\QEDA}{\hfill\ensuremath{\blacksquare}}
%% Declaring title and author
%	the institute's logo
%\renewcommand{\mylogo}{\includegraphics[width=4.7in]{mpidr_logo_colour_en}}
\usepackage{color}
\definecolor{mygray}{rgb}{0.8,0.8,0.8}
\definecolor{yellow}{rgb}{1,1,0}

\defbeamertemplate{description item}{align left}{\insertdescriptionitem\hfill}
%%	should be the very last package to be loaded
\usepackage{hyperref}

%%%%%%%%%%%%%%%%%%%%%%%%%%%%%%%%%%
%%	Beginning of the document		%%
%%%%%%%%%%%%%%%%%%%%%%%%%%%%%%%%%%
\begin{document}

%%	titlepage - fixed frame:
%%	========================

% \begin{frame}
% 	\titlepage
% \end{frame}
\begin{frame}[plain]
	%\titlepage
	\vspace{-3cm}
 \centerline{\includegraphics[scale=.165]{beamerstrip3.png}}

	
	\huge
	\vspace{1em}
	
	Alignment, clocking, and macro patterns of episodes in the life course\\
	\vspace{1em}
	\large 
	Tim Riffe 
\end{frame}
%-------------------

% motivation:
% 1) late-life sequence analysis ending in death (pattern detection)
% 2) matrix expression for average number of episodes (tenure statistics)
\begin{frame}[plain]
\Large
\centering
 2 motivating observations:\vspace{2em}
 \begin{enumerate}[<+->]
 \item sequence analysis of trajectories ending in death \only<3-4>{\textcolor{red}{(pattern detection)}} \only<4>{(Y. Hu)}
 \item matrix expression for average episode count \only<3-4>{\textcolor{red}{(tenure statistics)}} \only<4>{(C. Dudel)}
 \end{enumerate}
 % get screenshot from June, also paper from Dudel in email.
\end{frame}
%

\begin{frame}[plain]
\Large
\centering
 2 motivating questions:\vspace{2em}
 \begin{enumerate}
 \item would different patterns emerge if trajectories were aligned on moment of death?
 \item what is the age pattern of average episode duration?
 \end{enumerate}
 % get screenshot from June, also paper from Dudel in email.
\end{frame}
%

\begin{frame}[plain]
\Large
\centering
 2 \emph{procedural} solutions:\vspace{2em}
 \begin{enumerate}
 \item trajectory shifting to state transitions: \only<2>{\textcolor{red}{alignment}}
 \item flexible episode recording: \only<2>{\textcolor{red}{clocking}}
 \end{enumerate}
 % get screenshot from June, also paper from Dudel in email.
\end{frame}
%

\begin{frame}[plain]
\Large
\centering
An illustration
\pause
\begin{itemize}[<+->]
  \item Take transition matrix from \normalsize{Dudel \& Myrskl\"a (2017)}.
  \item Simulate trajectories using \texttt{rmarkovchain()} in \texttt{markovchain}
  package \normalsize{(Spedicato, 2017)}.
  \item Demonstrate concepts of \textcolor{red}{alignment} and \textcolor{red}{clocks}
  \item Generate (stationary) novel macro patterns
\end{itemize}
\end{frame}

%\begin{frame}[plain]
%\Large
%\centering
%A visual explanation
%
%\only<1>{\hspace{-8em}\includegraphics[width=\textwidth, keepaspectratio]{Figures/LifeLineSeq/chron1.pdf}}
%\only<2>{\hspace{-8em}\includegraphics[width=\textwidth, keepaspectratio]{Figures/LifeLineSeq/chron2.pdf}}
%\only<3>{\hspace{-8em}\includegraphics[width=\textwidth, keepaspectratio]{Figures/LifeLineSeq/chron3.pdf}}
%\only<4>{\hspace{-8em}\includegraphics[width=\textwidth, keepaspectratio]{Figures/LifeLineSeq/chron4.pdf}}
%\only<5>{\hspace{-8em}\includegraphics[width=\textwidth, keepaspectratio]{Figures/LifeLineSeq/chron5.pdf}}
%\only<6>{\hspace{-8em}\includegraphics[width=\textwidth, keepaspectratio]{Figures/LifeLineSeq/than1.pdf}}
%\only<7>{\includegraphics[width=\textwidth, keepaspectratio]{Figures/LifeLineSeq/together.pdf}}
%\end{frame}
%
%


%\begin{frame}[plain]
%\Large
% \begin{block}{Brouard-Carey Symmetry}
%  The age distribution is identical to / symmetrical with the time-to-death distribution.
% \end{block}
%\end{frame}
%
%\begin{frame}[plain]
%\Large
% \begin{block}{Transient Symmetry}
%  Within a given \textcolor{OliveGreen}{state}, the \textcolor{red}{time-spent}
%  distribution is equal to the \textcolor{blue}{time-to-exit} distribution.
% \end{block}
%\end{frame}
%
%\begin{frame}[plain]
%\Large
% \begin{block}{Transient equality}
%  Under stationarity, the probability that a randomly selected individual is in
%  state \textcolor{OliveGreen}{$\mathbf{s}$} and entered
%  \textcolor{OliveGreen}{$\mathbf{s}$} \textcolor{red}{$\mathbf{x}$} years ago
%  is equal to the probability of being in state
%  \textcolor{OliveGreen}{$\mathbf{s}$} and exiting in
%  \textcolor{blue}{$\mathbf{x}$} years.
% \end{block}
%\end{frame}
%
%\begin{frame}[plain]
%\Large
%\centering
%Requisites:
%\begin{itemize}
%\item All vital and state transition schedules fixed.
%\item No growth (births = deaths).
%\end{itemize}
%\end{frame}
%
%\begin{frame}[plain]
%\Large
%\centering
%Probabilistic result:
%\begin{itemize}[<+->]
%\item The expected age-state structure is frozen.
%\item Each potential discrete state trajectory has a fixed probability of
%occurring.
%\item Same for past and future cohorts.
%\end{itemize}
%\end{frame}
%
%\begin{frame}[plain]
%\Large
%\centering
%Deterministic result (problematic):
%\begin{itemize}[<+->]
%\item The age-state structure is frozen.
%\item Each discrete state trajectory occurs for a fixed fraction of a birth
%cohort.
%\item Same for past and future cohorts.
%\end{itemize}
%\end{frame}
%
%\begin{frame}[plain]
%\Large
%\centering
%Deterministic result (friendly):
%\begin{itemize}[<+->]
%\item The age-state structure is frozen.
%\item The same finite set of discrete state trajectories.
%\item Same for past and future cohorts (all clones).
%\end{itemize}
%\end{frame}
%
%\begin{frame}[plain]
%\Large
%\centering
%Prove for friendly case, generalize to probability case.
%\end{frame}
%
%\begin{frame}[plain]
%\Large
%Step 1
%\begin{overlayarea}{\textwidth}{.4\textheight}
%\begin{center}
%\only<1>{
%$A^{(i)}=\left\{ 0\right\}$\\
%%\includegraphics[scale = 1]{Figures/SingleLifeAnim1/step0.pdf}\\
%$T^{(i)}=\left\{ \tau_1 \right\}$
%}
%\only<2>{
%$A^{(i)}=\left\{0, a_1 \right\}$\\
%%\includegraphics[scale = 1]{Figures/SingleLifeAnim1/step1.pdf}\\
%$T^{(i)}=\left\{ \tau_1 \right\}$
%}
%\only<3>{
%$A^{(i)}=\left\{0, a_1,a_2 \right\}$\\
%%\includegraphics[scale = 1]{Figures/SingleLifeAnim1/step2.pdf}\\
%$T^{(i)}=\left\{ \tau_1,\tau_2 \right\}$
%}
%\only<4>{
%$A^{(i)}=\left\{0, a_1,a_2,a_3 \right\}$\\
%%\includegraphics[scale =
%1]{Figures/SingleLifeAnim1/step3.pdf}\\
%$T^{(i)}=\left\{ \tau_1,\tau_2,\tau_3 \right\}$} 
%\only<5>{
%$A^{(i)}=\left\{0, a_1,a_2,a_3,\ldots, a_{K-1} \right\}$\\
%%\includegraphics[scale = 1]{Figures/SingleLifeAnim1/step14.pdf}\\
%$T^{(i)}=\left\{ \tau_1,\tau_2,\tau_3, \ldots, \tau_{K-1} \right\}$
%}
%\only<6>{
%$A^{(i)}=\left\{0, a_1,a_2,a_3,\ldots, a_{K-1}, a_{K} \right\}$\\
%%\includegraphics[scale = 1]{Figures/SingleLifeAnim1/step15.pdf}\\
%$T^{(i)}=\left\{ \tau_1,\tau_2,\tau_3, \ldots, \tau_{K-1}, 0 \right\}$
%}
%\only<7>{
%$A^{(i)}=\left\{0, a_1,a_2,a_3,\ldots, a_{K-1}, a_{K} \right\}$\\
%%\includegraphics[scale = 1]{Figures/SingleLifeAnim1/stepsmalldelta.pdf}\\
%$T^{(i)}=\left\{ \tau_1,\tau_2,\tau_3, \ldots, \tau_{K-1}, 0 \right\}$
%}
%\end{center}
%\end{overlayarea}
%\end{frame}
%
%--------------------------------------------------------


%
%\begin{frame}[plain]
%\Large
%\begin{center}
%Implications:
%\pause
%\begin{itemize}[<+->]
%  \item Equal time spent-left distributions within states.
%  \item Within episode \emph{order}. 
%  \item Cumulative over episodes.
%  \item Conditional on anything (age, TTD).
%  \item Brouard-Carey is degenerate case.
%\end{itemize}
%\end{center}
%\end{frame}
%
% Now we install the new template for the following frames:
%{
%\usebackgroundtemplate{%
% 
  % \includegraphics[width=\paperwidth,height=\paperheight]{Figures/eberhard-grossgasteiger-410620-unsplash.jpg}} \begin{frame}[plain]
%\tiny
%\flushright
%\vspace{18cm}
%   Photo by Eberhard Grossgasteiger on Unsplash
%\end{frame}
%}
% Now we install another template, effective from now on:
%\begin{frame}[plain]
%\Large
%\begin{center}
%Simulate
%\begin{itemize}[<+->]
%  \item Take transition matrix from \normalsize{Dudel \& Myrskl\"a (2017)}
%  \item Simulate trajectories; \texttt{rmarkovchain()} in \texttt{markovchain}
%  package.
%  \item Take census in stationary series. 
%  \item Assume observation at half interval.
%  \item Tabulate time spent and left in sampled episodes.
%  \item Compare distributions.
%\end{itemize}
%\end{center}
%\end{frame}
%
%\begin{frame}[plain]
%\Large
%\begin{center}
%\includegraphics[height=\paperheight]{Figures/SimResults.pdf}
%\end{center}
%\end{frame}


%% this puts image as slide background
%{
%\usebackgroundtemplate{%
%  \includegraphics[width=\paperwidth,height=\paperheight]{Figures/eberhard-grossgasteiger-410620-unsplash.jpg}} 
%\begin{frame}
%\vspace{9cm}
%\Large
%\begin{center}
%Estimate from the reflection!\\ Thanks! \\
%riffe@demogr.mpg.de
%\end{center}
%\end{frame}
%}
%
%%%%%%%%%%%%%%%%%%%%%%%%%%%%%%%%%%
%%	End of the document			%%
%%%%%%%%%%%%%%%%%%%%%%%%%%%%%%%%%%
\end{document}



% Photo by eberhard grossgasteiger on Unsplash






